\RequirePackage[l2tabu, orthodox]{nag}
\documentclass[12pt]{article}
\usepackage[utf8]{inputenc} 
\usepackage[T1]{fontenc}
\usepackage[english]{babel} 
\usepackage[margin=2.5cm]{geometry}
\geometry{a4paper}
\usepackage{longtable}
\usepackage{subfigure}
\usepackage[normalem]{ulem}
\usepackage{cleveref}
\usepackage{tabularx}

\usepackage{array,arydshln}
%\setlength\dashlinedash{0.2pt}
%\setlength\dashlinegap{4.5pt}
%\setlength\arrayrulewidth{0.2pt}
\newcolumntype{C}[1]{>{\centering\arraybackslash}m{#1}}
\newcolumntype{R}[1]{>{\raggedleft\arraybackslash}m{#1}}

\usepackage{multicol}
\usepackage[table]{xcolor}
\usepackage{todonotes}
\usepackage{menukeys}
\usepackage{listings}
%----------------------------KODE START---------------------------------------
%\usepackage{listings}
%\usepackage{color}
%\usepackage[usenames,dvipsnames,table]{xcolor}
%\definecolor{gray}{rgb}{0.5,0.5,0.5}
%\definecolor{mauve}{rgb}{0.58,0,0.82}
%\lstset{
%  basicstyle=\footnotesize,
%  numbers=left,
%  numberstyle=\tiny\color{gray},
%  stepnumber=1,
%  numbersep=10pt,
%  backgroundcolor=\color{white},
%  showspaces=false,               % show spaces adding particular underscores
%  showstringspaces=false,         % underline spaces within strings
%  showtabs=false,                 % show tabs within strings adding particular underscores
%  frame=single,                   % adds a frame around the code
%  rulecolor=\color{black},        
%  tabsize=4,
%  captionpos=b,                   % sets the caption-position to bottom
%  breaklines=true,                % sets automatic line breaking
%  breakatwhitespace=false,        % sets if automatic breaks should only happen at whitespace
%  title=\lstname,                   % show the filename of files included with \lstinputlisting;
                                  % also try caption instead of title
%  keywordstyle=\color{mauve},          % keyword style
%  commentstyle=\color{Maroon},       % comment style
%  stringstyle=\color{BlueViolet},         % string literal style
%  escapeinside={\%*}{*)},            % if you want to add LaTeX within your code
%  morekeywords={*,...},              % if you want to add more keywords to the set
%  deletekeywords={...}              % if you want to delete keywords from the given language
%}
%----------------------------KODE SLUT----------------------------------------


%\setlength\parindent{0pt} % Makes \noindent standard
\usepackage{graphicx} 
\usepackage{sidecap}
\usepackage{caption}
%\usepackage{subcaption}
\usepackage{float} 
\usepackage{wrapfig} % Allows in-line images if needed
\usepackage{hyperref}
\usepackage{amsmath}
\usepackage{amsfonts}
\usepackage{mathtools}
\hypersetup{colorlinks=false,hidelinks, citecolor=black, urlcolor=black}
\usepackage{csquotes}
\usepackage{comment}
\usepackage{mathtools}
\DeclarePairedDelimiter{\ceil}{\lceil}{\rceil}

\usepackage[dot, autosize, outputdir="dotgraphs/"]{dot2texi}
\usepackage{tikz}
\usetikzlibrary{shapes}
\usepackage{url}
\usepackage{booktabs}
\usepackage{multirow}
\usepackage{longtable}
\setcounter{secnumdepth}{4}
\setcounter{tocdepth}{4}
\usepackage[titletoc]{appendix} % Names appendices "Appendix A"
                                % instead of just A in Contents
\usepackage[bottom]{footmisc}
\usepackage{pdfpages}
\usepackage{algorithm}% http://ctan.org/pkg/algorithms
\usepackage{algpseudocode}% http://ctan.org/pkg/algorithmicx


%\usepackage{lmodern}
\usetikzlibrary{arrows,automata}
\usepackage{verbatim}

\linespread{1.2} 
\graphicspath{{./figures/}} 

% fancy drawings
\usepackage{pgf}
\usepackage{epigraph}

% \epigraphsize{\small}% Default
\setlength\epigraphwidth{8cm}
\setlength\epigraphrule{0pt}

\usepackage{etoolbox}

\makeatletter
\patchcmd{\epigraph}{\@epitext{#1}}{\itshape\@epitext{#1}}{}{}
\makeatother

%\usepackage{boxproof}
%\usepackage{nomencl}
\usepackage{natbib}

\newcommand{\fasto}{\textsc{Fasto} }
\newcommand{\mips}{\textsc{Mips} }
\newcommand{\mars}{Mars }
\makeatletter
\def\BState{\State\hskip-\ALG@thistlm}
\makeatother

%-----------------------------------------------------------------------------
% HEADER AND FOOTER STUFF
%-----------------------------------------------------------------------------
\usepackage{fancyhdr}
\usepackage{lastpage} % Making it possible to write ``Page x of y'' in the footer

\pagestyle{fancy}
\fancyhf{}
% Header stuff below
\lhead{Jenny-Margrethe Vej} 
\chead{}
\rhead{rwj935} 
% Footer stuff below
\cfoot{Page \thepage \hspace{1pt} of \pageref{LastPage}} % To the left at the bottom

%-----------------------------------------------------------------------------
\begin{document}

\includepdf[pages={-}]{../forside_tex_ting/forside.pdf}

\begin{abstract} 
  
\end{abstract}

%Fra Torbens slide: 
%Et resumé (abstract) er
%En uhyre kort (5-20 linjer), præcis, kvantitativ beskrivelse af resultaterne i rapporten.
%Kun resultater! Metodik skal kun medtages, hvis den er relevant for at fortolke resultaterne. Alt andet er ligegyldigt
%Tænk: Hvis en meget travl beslutningstager (institutleder/direktøren/Torben) skal beslutte, om dokumentet er hans tid værd, skal vedkommende kunne afgøre det fra resuméet.
%Remember to make the abstract both in danish and english - english first\\

\newpage
\tableofcontents
%\nocite{*}

\newpage
\renewcommand{\abstractname}{Acknowledgements}
\begin{abstract}
\end{abstract}

\newpage  
\listoffigures
\addcontentsline{toc}{section}{List of Figures} %\caption[short caption for lof/lot]{long figure/table caption}

\listoftables
\addcontentsline{toc}{section}{List of Tables} %\caption[short caption for lof/lot]{long figure/table caption}

\newpage
\section{Introduction}

\section{Previous / Related Work}
Something about QoL + the data from Kates lab in Geneva and the data from DTU? Or do I create a chapter for each of them alone? 

\section{Design of Experiment}
Something general about the experiment and target group, and then some details (subchapters?) about the participants, the watches, the app to collect the data from the smartphones with, my initial surveys etc. 

\section{The Experiment}
Something about how the experiment went. What challenges did I meet in those 4 weeks. What did I end up asking the participants about, etc. 

\section{Analyse of Experiment Results}
Write about accuracy, speed and privacy-awareness/violation of a solution. Detect anomalies and unhealthy sleeping patterns. 

\section{Algorithm for Sleep Assessment}
Find hypothesises for a digital solution to help combat the unhealthy sleeping patterns, and explain the algorithm and the work behind it. 

\section{Evaluation of Algorithm}
Something about the second experiment in April. How did it go, what was the results regarding the algorithm etc. 

\section{Design Implications}
Provide design implications for a digital solution for mobilephones to help combat unhealthy behaviour in students sleeping patterns

\section{Conclusion}


\newpage
\bibliographystyle{plainnat}
%\bibliographystyle{unsrt}
\bibliography{bibliography}

%\newpage
%\appendix
%\section{Appendix}


\end{document}