\RequirePackage[l2tabu, orthodox]{nag}
\documentclass[12pt]{article}
\usepackage[utf8]{inputenc} 
\usepackage[T1]{fontenc}
\usepackage[english]{babel} 
\usepackage[margin=2.5cm]{geometry}
\geometry{a4paper}
\usepackage{longtable}
\usepackage{subfigure}
\usepackage[normalem]{ulem}


\usepackage{array,arydshln}
%\setlength\dashlinedash{0.2pt}
%\setlength\dashlinegap{4.5pt}
%\setlength\arrayrulewidth{0.2pt}
\newcolumntype{C}[1]{>{\centering\arraybackslash}m{#1}}
\newcolumntype{R}[1]{>{\raggedleft\arraybackslash}m{#1}}

\usepackage{multicol}
\usepackage[table]{xcolor}
\usepackage{todonotes}
\usepackage{menukeys}

%----------------------------KODE START---------------------------------------
%\usepackage{listings}
%\usepackage{color}
%\usepackage[usenames,dvipsnames,table]{xcolor}
%\definecolor{gray}{rgb}{0.5,0.5,0.5}
%\definecolor{mauve}{rgb}{0.58,0,0.82}
%\lstset{
%  basicstyle=\footnotesize,
%  numbers=left,
%  numberstyle=\tiny\color{gray},
%  stepnumber=1,
%  numbersep=10pt,
%  backgroundcolor=\color{white},
%  showspaces=false,               % show spaces adding particular underscores
%  showstringspaces=false,         % underline spaces within strings
%  showtabs=false,                 % show tabs within strings adding particular underscores
%  frame=single,                   % adds a frame around the code
%  rulecolor=\color{black},        
%  tabsize=4,
%  captionpos=b,                   % sets the caption-position to bottom
%  breaklines=true,                % sets automatic line breaking
%  breakatwhitespace=false,        % sets if automatic breaks should only happen at whitespace
%  title=\lstname,                   % show the filename of files included with \lstinputlisting;
                                  % also try caption instead of title
%  keywordstyle=\color{mauve},          % keyword style
%  commentstyle=\color{Maroon},       % comment style
%  stringstyle=\color{BlueViolet},         % string literal style
%  escapeinside={\%*}{*)},            % if you want to add LaTeX within your code
%  morekeywords={*,...},              % if you want to add more keywords to the set
%  deletekeywords={...}              % if you want to delete keywords from the given language
%}
%----------------------------KODE SLUT----------------------------------------


\setlength\parindent{0pt} % Makes \noindent standard
\usepackage{graphicx} 
\usepackage{sidecap}
\usepackage{caption}
%\usepackage{subcaption}
\usepackage{float} 
\usepackage{wrapfig} % Allows in-line images if needed
\usepackage{hyperref}
\usepackage{amsmath}
\usepackage{amsfonts}
\usepackage{mathtools}
\hypersetup{colorlinks=false,hidelinks, citecolor=black, urlcolor=black}
\usepackage{csquotes}
\usepackage{comment}
\usepackage{mathtools}
\DeclarePairedDelimiter{\ceil}{\lceil}{\rceil}

\usepackage[dot, autosize, outputdir="dotgraphs/"]{dot2texi}
\usepackage{tikz}
\usetikzlibrary{shapes}
\usepackage{url}
\usepackage{booktabs}
\usepackage{multirow}
\usepackage{longtable}
\setcounter{secnumdepth}{4}
\setcounter{tocdepth}{4}
\usepackage[titletoc]{appendix} % Names appendices "Appendix A"
                                % instead of just A in Contents
\usepackage[bottom]{footmisc}
\usepackage{pdfpages}
\usepackage{algorithm}% http://ctan.org/pkg/algorithms
\usepackage{algpseudocode}% http://ctan.org/pkg/algorithmicx


%\usepackage{lmodern}
\usetikzlibrary{arrows,automata}
\usepackage{verbatim}

\linespread{1.2} 
\graphicspath{{./figures/}} 

% fancy drawings
\usepackage{pgf}
\usepackage{epigraph}

% \epigraphsize{\small}% Default
\setlength\epigraphwidth{8cm}
\setlength\epigraphrule{0pt}

\usepackage{etoolbox}

\makeatletter
\patchcmd{\epigraph}{\@epitext{#1}}{\itshape\@epitext{#1}}{}{}
\makeatother

%\usepackage{boxproof}
%\usepackage{nomencl}
\usepackage{natbib}

\newcommand{\fasto}{\textsc{Fasto} }
\newcommand{\mips}{\textsc{Mips} }
\newcommand{\mars}{Mars }
\makeatletter
\def\BState{\State\hskip-\ALG@thistlm}
\makeatother

%-----------------------------------------------------------------------------
% HEADER AND FOOTER STUFF
%-----------------------------------------------------------------------------
\usepackage{fancyhdr}
\usepackage{lastpage} % Making it possible to write ``Page x of y'' in the footer

%\pagestyle{fancy}
%\fancyhf{}
% Header stuff below
%\lhead{Jenny-Margrethe Vej} 
%\chead{}
%\rhead{rwj935} 
% Footer stuff below
%\cfoot{Page \thepage \hspace{1pt} of \pageref{LastPage}} % To the left at the bottom

%-----------------------------------------------------------------------------
\begin{document}
%\begin{titlepage}

\begin{center}
\textsc{\Large Smartphone Perspective on Sleeping Patterns}\\[0.5cm] 
\textsc{\large Project description}\\[0.5cm] 

%\vfill

\emph{Author:}
\\
Jenny-Margrethe \textsc{Vej} - rwj935\\ 
\vspace{10mm}

{\large \today}\\[3cm] 
\end{center}
\vspace{-20mm}

\section{Problem Statement}
First part of this project will be to research on sleep patterns in the data collected by Social 
Fabric\cite{Stopczynski2014}. This research will help us get an understanding on what kind of 
sleep data is possible to collect ourself, and we will have some data to compare with. After this 
we will conduct our own experiment investigating the sleeping pattern of students using smartphones. 
With the data from this investigation we will examine how Digital Cognitive Behavioural Therapy 
(DCBT) can be embedded in a digital solution for mobilephones to help combat unhealthy behaviour in 
students sleeping patterns. We expect the digital solution to be an application, and we will evaluate
this through experiments and interviews. 

\section{Motivation}
It is widely known that sleep is essential \cite{Gumbiner2012}, and sleep deprivation affects 
us in many ways. It is also commonly known that students tend to skip a few hours of sleep to 
get more out of the day. But skipping hours of sleep will most likely not do anything good - instead
students get poorer performance in school \cite{Gilbert2010}. \\

According to the central authority on Danish statistics, Statistics Denmark, every household in 
Denmark owned at least one mobilephone in 2013 \cite{DanmarksStatistik2013}, and according 
to a article in Dailymail that same year \cite{Woollaston2013}, we check our phone 110 times a 
day and up to every 6 seconds in the evening. \\

Knowing that almost everyone owns a mobilephone, and students tend to skip sleep to work more,
we thought it would be great to investigate how we can use the mobilephone to improve our 
sleeping patterns. 

\section{Elaboration}
Using already existing apps for mobilephones we will investigate students sleeping patterns over a 
period of time of four weeks. While collecting data we will concurrently evaluate the experiment through 
questionnaires and interviews with the participants. The collected data will be analysed to see if we
can create a digital solution for mobilephones to help combat unhealthy behaviour in students sleeping
patterns. \\

The digital solution created will be evaluated in another period of time of four weeks, hopefully with 
the same students that participated in the first experiment. If it is not possible to get the same 
participants, we will have other students with a similar profile participating instead. The project will be 
conducted in collaboration with a psychologist specialised in sleep. 

\section{Learning Goals}
	\begin{itemize}
		\item Describe and select relevant data from existing studies
		\item Design and execute an experiment for collecting data on sleeping patters of students
		\item Analyse sleeping patterns in order to detect anomalies and unhealthy patterns 
		\item Design a digital solution for mobilephones to help combat unhealthy behaviour in students sleeping patterns
		\item Evaluate the digital solution through experiments and interviews
	\end{itemize}
	
\section{Time schedule}
Preliminary time schedule for the project:
	\begin{itemize}
		\item \textbf{Litterature study} - 2 months (including August)
			\begin{itemize}
				\item Study existing literature on sleeping behavior
				\item Select relevant data from existing studies (especially Social Fabric)
			\end{itemize}	
		\item \textbf{Set up experiment} - 1 month
			\begin{itemize}
				\item Find participants
				\item Decide on how to measure the data needed
				\item Look into small pilot studies to test different applications for measurements
			\end{itemize}	
		\item \textbf{Conduct experiment} - 1 month
			\begin{itemize}
				\item Suggested period of time: November 4, 2015 to December 2, 2015
			\end{itemize}	
		\item \textbf{Analyse results of experiment} - 1 month
			\begin{itemize}
				\item Detect anomalies and unhealthy sleeping patterns
				\item Find hypothesises for a digital solution to help combat the unhealthy sleeping 
				patterns   
			\end{itemize}	
		\item \textbf{Design the digital solution} - 3 months 
			\begin{itemize}
				\item Design and develop digital solution
				\item Pilot tests on solution 
			\end{itemize}	
		\item \textbf{Conduct experiment with digital solution} - 1 month
			\begin{itemize}
				\item Suggested period of time: March 30, 2016 to April 27, 2016
			\end{itemize}	
		\item \textbf{Write project report} - concurrently with all other phases + 1.5 months
			\begin{itemize}
				\item Deadline for project is July 15, 2016, but I will try to have the report done for proofreading around June 20, 2016 
			\end{itemize}	
	\end{itemize}

%\bibliographystyle{plainnat}
\bibliographystyle{unsrt}
\bibliography{bibliography}


%-----------------------------------------------------------------------------
\end{document}